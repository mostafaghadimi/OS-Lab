\سؤال{پیاده‌سازی ضرب ماتریسی ۴ در ۴ به کمک \lr{SSE}}

در این پروژه سعی کرده‌ایم تا ضرب ماتریسی را به کمک دستورالعمل‌های \lr{SSE} انجام داده و آن را با ضرب ماتریسی معمولی مقایسه کرده و نتایج و مراحل پیاده‌سازی را گزارش کنیم.

\begin{itemize}
	\item \textbf{ضرب معمولی ماتریسی}
	\begin{Verbatim}[tabsize=4]
void matmult_ref(Mat44 &out, const Mat44 &A, const Mat44 &B)
{
	Mat44 t; // write to temp
	for (int i=0; i < 4; i++) {
		for (int j=0; j < 4; j++) {
			t.m[i][j] = 0;
			t.m[i][j] += A.m[i][0]*B.m[0][j] 
			t.m[i][j] += A.m[i][1]*B.m[1][j]
			t.m[i][j] += A.m[i][2]*B.m[2][j]
			t.m[i][j] += A.m[i][3]*B.m[3][j];
		}
	}
	out = t;
}
	\end{Verbatim}
	\item به دست آوردن ترکیب خطی به کمک دستورات \lr{SSE}
	
	در این قسمت در تلاش هستیم تا درایه‌های ماتریس اول را در ردیف‌های ماتریس دوم ضرب کنیم.
	\begin{Verbatim}[tabsize=4]
static inline __m128 lincomb_SSE(const __m128 &a, const Mat44 &B)
{
	__m128 result;
	result = _mm_mul_ps(_mm_shuffle_ps(a, a, 0x00), B.row[0]);
	result = _mm_add_ps(result, _mm_mul_ps(_mm_shuffle_ps(a, a, 0x55), B.row[1]));
	result = _mm_add_ps(result, _mm_mul_ps(_mm_shuffle_ps(a, a, 0xaa), B.row[2]));
	result = _mm_add_ps(result, _mm_mul_ps(_mm_shuffle_ps(a, a, 0xff), B.row[3]));
	return result;
}
	\end{Verbatim}
	\item انجام ضرب به کمک \lr{SSE}
	\begin{Verbatim}[tabsize=4]
void matmult_SSE(Mat44 &out, const Mat44 &A, const Mat44 &B)
{
	__m128 out0x = lincomb_SSE(A.row[0], B);
	__m128 out1x = lincomb_SSE(A.row[1], B);
	__m128 out2x = lincomb_SSE(A.row[2], B);
	__m128 out3x = lincomb_SSE(A.row[3], B);
	
	out.row[0] = out0x;
	out.row[1] = out1x;
	out.row[2] = out2x;
	out.row[3] = out3x;
}
	\end{Verbatim}
	\item تست و آزمون کارکرد 
		\begin{itemize}
			\item اجرای ضرب معمولی ماتریسی
			\begin{Verbatim}[tabsize=4]
static void run_ref(Mat44 *out, const Mat44 *A, const Mat44 *B, int count)
{
	for (int i=0; i < count; i++)
	{
		int j = i & the_mask;
		matmult_ref(out[j], A[j], B[j]);
	}
}

			\end{Verbatim}
			\item
			اجرای ضرب به کمک \lr{SSE}
			\begin{Verbatim}[tabsize=4]
static void run_SSE(Mat44 *out, const Mat44 *A, const Mat44 *B, int count)
{
	for (int i=0; i < count; i++)
	{
		int j = i & the_mask;
		matmult_SSE(out[j], A[j], B[j]);
	}
}
			\end{Verbatim}
		\end{itemize}
	\item اجرای کد
	\begin{Verbatim}[tabsize=4]
int main(int argc, char **argv)
{
	static const struct {
		const char *name;
		void (*matmult)(Mat44 &out, const Mat44 &A, const Mat44 &B);
	} variants[] = {
		{ "ref",      matmult_ref },
		{ "SSE",      matmult_SSE },
	};
	static const int nvars = (int) (sizeof(variants) / sizeof(*variants));
	
	srand(1234);
	
	// correctness tests
	for (int i=0; i < 1000000; i++)
	{
		Mat44 A, B, out, ref_out;
		randmat(A);
		randmat(B);
		matmult_ref(ref_out, A, B);
		
		for (int j=0; j < nvars; j++)
		{
			variants[j].matmult(out, A, B);
			if (memcmp(&out, &ref_out, sizeof(out)) != 0)
			{
				fprintf(stderr, "%s fails test\n", variants[j].name);
				exit(1);
			}
		}
	}
	
	printf("all ok.\n");
	
	// performance tests
	
	static const struct {
		const char *name;
		void (*run)(Mat44 *out, const Mat44 *A, const Mat44 *B, int count);
	} perf_variants[] = {
		{ "ref",      run_ref },
		{ "SSE",      run_SSE },
	};
	static const int nperfvars = (int) (sizeof(perf_variants) / sizeof(*perf_variants));
	
	/* 
	test done on my laptop with following results:
	ref: 59.00 cycles
	SSE: 20.52 cycles
	*/
	
	Mat44 Aperf, Bperf, out;
	randmat(Aperf);
	randmat(Bperf);
	
	for (int i=0; i < nvars; i++)
	{
		static const int nruns = 4096;
		static const int muls_per_run = 4096;
		unsigned long long best_time = ~0ull;
		
		for (int run=0; run < nruns; run++)
		{
			unsigned long long time = __rdtsc();
			perf_variants[i].run(&out, &Aperf, &Bperf, muls_per_run);
			time = __rdtsc() - time;
			if (time < best_time)
				best_time = time;
		}
		
		double cycles_per_run = (double) best_time / (double) muls_per_run;
		printf("%12s: %.2f cycles\n", perf_variants[i].name, cycles_per_run);
	}
	
	return 0;
}

	\end{Verbatim}
\end{itemize}