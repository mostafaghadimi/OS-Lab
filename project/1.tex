\سؤال{پیاده‌سازی ضرب ماتریسی ۴ در ۴ به کمک \lr{SSE}}

در این پروژه سعی کرده‌ایم تا ضرب ماتریسی را به کمک دستورالعمل‌های \lr{SSE} انجام داده و آن را با ضرب ماتریسی معمولی مقایسه کرده و نتایج و مراحل پیاده‌سازی را گزارش کنیم.

\begin{itemize}
	\item \textbf{ضرب معمولی ماتریسی}
	\begin{Verbatim}[tabsize=4]
void matmult_ref(Mat44 &out, const Mat44 &A, const Mat44 &B)
{
	Mat44 t; // write to temp
	for (int i=0; i < 4; i++) {
		for (int j=0; j < 4; j++) {
			t.m[i][j] = 0;
			t.m[i][j] += A.m[i][0]*B.m[0][j] 
			t.m[i][j] += A.m[i][1]*B.m[1][j]
			t.m[i][j] += A.m[i][2]*B.m[2][j]
			t.m[i][j] += A.m[i][3]*B.m[3][j];
		}
	}
	out = t;
}
	\end{Verbatim}
	\item به دست آوردن ترکیب خطی به کمک دستورات \lr{SSE}
	
	در این قسمت در تلاش هستیم تا درایه‌های ماتریس اول را در ردیف‌های ماتریس دوم ضرب کنیم.
	\begin{Verbatim}[tabsize=4]
static inline __m128 lincomb_SSE(const __m128 &a, const Mat44 &B)
{
	__m128 result;
	result = _mm_mul_ps(_mm_shuffle_ps(a, a, 0x00), B.row[0]);
	result = _mm_add_ps(result, _mm_mul_ps(_mm_shuffle_ps(a, a, 0x55), B.row[1]));
	result = _mm_add_ps(result, _mm_mul_ps(_mm_shuffle_ps(a, a, 0xaa), B.row[2]));
	result = _mm_add_ps(result, _mm_mul_ps(_mm_shuffle_ps(a, a, 0xff), B.row[3]));
	return result;
}
	\end{Verbatim}
	\item انجام ضرب به کمک \lr{SSE}
	\begin{Verbatim}[tabsize=4]
void matmult_SSE(Mat44 &out, const Mat44 &A, const Mat44 &B)
{
	__m128 out0x = lincomb_SSE(A.row[0], B);
	__m128 out1x = lincomb_SSE(A.row[1], B);
	__m128 out2x = lincomb_SSE(A.row[2], B);
	__m128 out3x = lincomb_SSE(A.row[3], B);
	
	out.row[0] = out0x;
	out.row[1] = out1x;
	out.row[2] = out2x;
	out.row[3] = out3x;
}
	\end{Verbatim}
\end{itemize}